\documentclass[12pt,a4paper]{article} %(nie wiem kiedy) usunąłem tą linię i nie wiem czy tak było
%przykladowe pakiety
\usepackage[utf8]{inputenc}
\usepackage{polski}
\usepackage{graphicx}
\usepackage{amsmath} % w zasadzie tez mozna wywalic 
\usepackage{booktabs}
\usepackage{listings}
\usepackage{color}
\usepackage{geometry} 

\geometry{
	a4paper,
	left = 2.5cm,
	right = 2.5cm,
	top = 2cm,
	bottom = 2cm
}

\begin{document}

\begin{titlepage}

\newcommand{\HRule}{\rule{\linewidth}{0.5mm}}

\center
 
\textsc{\LARGE Politechnika Wrocławska}\\[1.5cm] 
\textsc{\Large Zastosowanie informatyki w gospodarce}\\[0.5cm]

\HRule \\[0.5cm]
{ \huge \bfseries Relacje pomiędzy bytami w tekstach literackich}\\[0.2cm]
\HRule \\[1.6cm]
 
 
\begin{minipage}{0.4\textwidth}
\begin{flushleft} \large
\emph{Lider grupy:}\\
Przemysław \textsc{Wujek} 234983\\
\emph{Skład grupy:}\\
Paweł \textsc{Czarnecki} 234974\\
Łukasz \textsc{Łupicki} 257536\\
Dawid \textsc{Piechota} 235851\\
Bartosz \textsc{Rodziewicz} 226105\\
Wojciech \textsc{Wójcik} 235621\\
\end{flushleft}
\end{minipage}
~
\begin{minipage}{0.4\textwidth}
\begin{flushright} \large
\emph{Prowadzący:} \\
dr inż. Tomasz \textsc{Walkowiak} 
\end{flushright}
\end{minipage}\\[4cm]

\vfill
{\large 04 kwietnia 2020}

\end{titlepage}
   
\newpage

\section{Wymagania}
    \subsection{Funkcjonalne}
        \begin{itemize}
            \item Wyświetlanie powiązań w formie grafów
            \item Wczytywanie tekstów utworów do analizy
            \item Parsowanie zawartości danych wyjściowych narzędzia Ner, aby wytworzyć bazę powiązań pomiędzy postaciami
            \item Grupowanie/filtrowanie/scalanie postaci z poziomu widoku grafu
            \item Podświetlanie powiązań między wybranymi postaciami
            \item Eksport grafów do plików graficznych
            \item Buforowanie wyników analizy poprzednich tekstów
        \end{itemize}

    \subsection{Niefunkcjonalne}
        \begin{itemize}
            \item System powinien być napisany jako aplikacja internetowa
            \item System powinien posiadać prosty interfejs użytkownika zgodny z obowiązującymi standardami na rynku
        \end{itemize}


\section{Technologie}
    \begin{itemize}
        \item Python, Flask
        \item JavaScript, framework do wybrania
    \end{itemize}

\section{Kamienie milowe}
    \subsection{16.04 - pierwszy kamień milowy, raport}
        \begin{itemize}
            \item Przetestowanie technologii prezentacji danych
            \item Porównanie wydajności technologii prezentacji danych
            \item Zapoznanie się z dokumentacją i przetestowanie narzędzia do analizy tekstu
            \item Opracowanie metody szukania zależności między postaciami
            \item Ustalenie akcji dostępnych w reprezentacji graficznej powiązań
        \end{itemize}

    \subsection{7.05 - drugi kamień milowy, raport i kod}
        \begin{itemize}
            \item Przygotowanie oprogramowania backendowego
                \begin{itemize}
                    \item wczytywanie tekstów z plików
                    \item dane dotyczące poszczególnych postaci w całym analizowanym tekście
                    \item określanie relacji pomiędzy bytami wykorzystując okna analizujące cały tekst
                    \item zapis wybranych wyszukań do plików graficznych
                \end{itemize}{}
            \item Implementacja przetestowanych rozwiązań do wyświetlania grafów
        \end{itemize}

    \subsection{27.05 - trzeci kamień milowy kamień, prezentacja/raport i kod}
        \begin{itemize}
            \item Przygotowanie prototypu części frontendowej przedstawiającego kluczowe funkcjonalności systemu
            \item Próba implementacji rozpoznawania rozdziałów, lub akapitów ustalających okno analizy
            \item Próba implementacji dyzambiguacji bytów po stronie backendu
            \item Zapis grafu do pliku, z którego następnie aplikacja wczyta graf
        \end{itemize}

    \subsection{11.06 - prezentacja i kod}
        \begin{itemize}
            \item Finalna wersja projektu
            \item Dokumentacja projektu
        \end{itemize}

\end{document}